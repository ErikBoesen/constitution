\documentclass{proc}

\begin{document}

\title{Team 1418 - Unified Organizational Constitution}
\author{Erik Boesen}

\maketitle

\section{Introduction and Motivations}
This document (hereinafter referred to as the "UOC") serves as a formalized outline of Team 1418's organizational structure and procedures. Team 1418 is a FIRST Robotics Competition team in Falls Church, VA, founded in 2004. Team 1418 is comprised of students from George Mason High School.

This document was created in the pre-season period of the 2019 FRC season. Prior to that point, Team 1418 had relied on an informal leadership structure. The UOC was initially designed, though is not limited, to several ends:
\begin{itemize}
  \item{To formalize a consistent strategy for appointing team captains and other leadership,}
  \item{To define a clear process for applications to the team, and create a procedural framework for handling membership, and}
  \item{To define duties and organizational responsibilities of team leadership, members, and mentors.}
\end{itemize}

\section{Meta}
\subsection{Amending}
This document may be amended through an affiramative vote of 3/4 of members as defined herein, with the approval of at least one mentor.

\section{Organization}
\subsection{Leadership}
We define several static leadership positions:
\begin{itemize}
  \item{Team Captain: This position can be occupied by one member at a time. With the approval of a mentor, the position can be scaled to a group of up to three captains. Nominated by mentor or voted upon.}
  \item{Head Programmer: To be occupied by the most qualified programmer. Nominated by Team Captain or voted upon.}
  \item{Head Mechanic: To be occupied by the most qualified mechanic. Nominated by Team Captain or voted upon.}
  \item{Head of Outreach: To be occupied by the most qualified member of the outreach subteam. Nominated by Team Captain or voted upon.}
\end{itemize}

\subsection{Duties of leadership}
\subsubsection{Team Captain}
The \textbf{Team Captain} acts as the voice of the team during promotional events, performing tasks such as speaking before the school board after the season. In addition to ceremonial roles, the team captain acts as the authority figure within the team, having the authority to make decisions on robot design or any other topics.

\subsubsection{Subteam Heads}
The \textbf{Head Programmer}, \textbf{Head Mechanic}, \textbf{Head of Outreach} each have the authority to make relevant decisions within their respective subteams.

\subsection{Drive Team}
The drive team must always contain at least one programmer and more than one mechanic.

When a driver graduates, new drivers should be as far from graduating as possible. This way, students will be able to remain in driving positions for multiple years, developing their skills and preparing to drive better when they're seniors. Seniors should not be selected as new drivers as they will graduate after a year, leaving the team with an inexperienced driver yet again.

\subsection{Applications}
Each year, students will be required to pass an application process, consisting of the following steps, to join the team:
\begin{itemize}
  \item{Passing raw application requirements: students must maintain at the very least a 3.0 unweighted high school GPA and be free of any major academic infractions in order to apply and remain on the team.}
  \item{Writing up to one page, summarizing any qualifications and explaining why they want to join the team and what they hope to contribute.}
  \item{If subteam leaders so desire, prospective subteam members may be made to submit additional application material such as a proficiency test.}
\end{itemize}
The team will not exceed 25 members in the 2019 season.

\end{document}
